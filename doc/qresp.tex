%\documentclass[reprint,prd, superscriptaddress, tightenlines, longbibliography, nofootinbib, eqsecnum, amsfonts, amsmath, floatfix, notitlepage, onecolumn]{revtex4-1}
\documentclass{article}
\usepackage[a4paper, total={7.1in, 9in}]{geometry}
\usepackage{amsmath} 
\usepackage{amssymb}

\usepackage{color}
\newcommand{\T}[0]{{\mathcal T}}
\newcommand{\Xd}[0]{{X^{\rm dat}}}
\newcommand{\Xwf}[0]{{X^{\rm WF}}}
\newcommand{\Cov}[0]{{\textrm{Cov}}}

\newcommand{\si}[0]{{s_{\rm i}}}
\newcommand{\ti}[0]{{t_{\rm i}}}
\newcommand{\ui}[0]{{u_{\rm i}}}
\newcommand{\vi}[0]{{v_{\rm i}}}
\newcommand{\so}[0]{{s_{\rm o}}}
\renewcommand{\to}[0]{{t_{\rm o}}}
\newcommand{\uo}[0]{{u_{\rm o}}}
\newcommand{\vo}[0]{{v_{\rm o}}}

\newcommand{\sr}[0]{{s_{\rm r}}}
\newcommand{\tr}[0]{{t_{\rm r}}}
\newcommand{\ur}[0]{{u_{\rm r}}}
\newcommand{\vr}[0]{{v_{\rm r}}}

\newcommand{\Ylm}[1]{\:_{#1}Y_{\ell m}}
\newcommand{\Ylms}[1]{\:_{#1}Y^*_{\ell m}}
\newcommand{\YLM}[1]{\:_{#1}Y_{L M}}
\newcommand{\YLMs}[1]{\:_{#1}Y^*_{L M}}
\newcommand{\av}[1]{\left\langle #1 \right\rangle}
\newcommand{\resp}{ {\mathcal R} }

\newcommand{\red}[1]{\color{red}#1 \color{black} }
\newcommand{\JC}[1]{\color{red}JC: #1\color{black}}
\newcommand{\hn}[0]{\hat n}
\begin{document}
\title{Notes on curved-sky QE responses etc.}
\maketitle
\tableofcontents
\vspace{1cm}
\JC{Document to be included with the pipeline release after submission of the revised L08.}

Lensing and others quadratic estimators used in \cite{??} are all built multiplying in position space spin transforms of spin-weighted fields. We may write all of these in the form \begin{equation}\label{QE}
 _{\so + \to}\hat d(\hn) \equiv  \left(\sum_{\ell m}\: w^{\so\si}_\ell \:_{\si} \bar X_{\ell m} \:_\so Y_{\ell m}(\hn)\right)\left(\sum_{\ell m}\:w^{\to\ti}_\ell  \:_{\ti} \bar X_{\ell m} \:_\to Y_{\ell m}(\hn)\right)
\end{equation}
where $\si, \ti$ are input spins, $\so, \to$ outputs spins, and $w_\ell^{\so\si}, w_\ell^{\ti\to}$ associated weights. The maps $_s \bar X_{lm}$ are the inverse variance filtered CMB maps,
\begin{equation}
	_0 \bar X_{\ell m} = -\bar T_{\ell m} , \quad _{\pm 2} \bar X_{\ell m} = -\left(\bar E_{\ell m} \pm i\bar B_{\ell m} \right).
\end{equation}
For purely analytical calculations, the filtering operation itself can be approximated as isotropic. For independently filtered temperature and polarization, the filtered $\bar T, \bar E, \bar B$ are directly proportional to $T, E$ and $B$ respectively. 
We keep the discussion focussed on generic fields $\bar X$ of arbitrary spins in the following. The gradient (G) and curl (C) modes of definite parity are defined through
\begin{eqnarray*}
		G^{s}_{LM} &= -\frac 12\left(\:_{|s|} d_{LM} + (-1)^s \:_{-|s|} d_{LM}\right)  \\
		C^{s}_{LM} &=-\frac 1{2i} \left( \:_{|s|} d_{LM} - (-1)^s \:_{-|s|} d_{LM} \right) .
\end{eqnarray*}
The formulae exposed here can be derived through simple application of this relation,
\begin{equation}
\sum_{m_1,m_2}\int d^2n\: \prod_{i = 1}^3\:_{s_i} Y_{\ell_i m_i}(\hn)\int d^2n'\: \prod_{i = 1}^3\:_{t_i} Y_{\ell_i m_i}(\hn') = \frac{2\ell_1 + 1}{4\pi}\frac{2\ell_2 + 1} {4\pi} 2\pi \int_{-1}^{1} d\mu \prod_{i = 1}^3 d^{\ell_i}_{s_i, t_i}(\mu)
\end{equation}
\subsection{(Semi-)analytical QE Gaussian noise bias.}
Q.E. noise (co)-variance can be evaluated very easily as was first demonstrated by Ref.~\cite{}. For two generic estimators as defined in Eq.~\eqref{QE}, we can jointly obtain their G and C co-variances with 4 one-dimensional integrals as we now describe.

Let $s = (\si, \so, w^{\si\so})$ collectively describes the in and out spins and weight function, and similarly for $t, u$ and $v$. Let the covariance function $N^{st, uv}_L$ be defined through
\begin{equation}
\begin{split}
 \delta_{LL'}\delta_{MM'}N_L^{stuv} &\equiv \left.\av{\:_{\so + \to}\hat d_{LM}\: _{\uo + \vo}\hat d^*_{L'M'}} \right|_{\rm Gauss}\\ &= (-1)^{\so + \to + \uo + \vo}2\pi  \int_{-1}^1 d \mu\:  d^L_{-\so - \to, -\uo - \vo}(\mu) \left[\xi^{su}(\mu)\:\xi^{tv}(\mu)  + \xi^{sv}(\mu)\:\xi^{tu}(\mu) \right]
\end{split}
\end{equation}
where $\xi$ are position-space correlation functions
\begin{equation}
\xi^{st}(\mu) \equiv  \sum_\ell \left(\frac{2\ell + 1}{4\pi}\right)w^{\so\si}_\ell w^{\to\ti}_\ell \bar C_\ell^{\si \ti} d^\ell_{\so,\to}(\mu)\textrm{ with } \bar C_\ell^{\si \ti} \equiv \av{ _{\si}\bar X_{\ell m}\: _{\ti} \bar X^*_{\ell m} }
\end{equation}
and $d^\ell_{mm'}$ are Wigner small d-matrices.
Then
\begin{equation} \boxed{
\begin{split}
\left.\av{\hat G^{\so + \to}_{LM} \hat G^{*, \uo + \vo}_{L' M'} }\right|_{\rm Gauss.} &=\delta_{LL'}\delta_{MM'} \frac 12\left[N_L^{stuv} +  (-1)^{\so + \to} N^{-s-tuv}_L\right] \\
		\left.\av{\hat C^{\so + \to}_{LM} \hat C^{*, \uo + \vo}_{L' M'} }\right|_{\rm Gauss.} &= \delta_{LL'}\delta_{MM'}\frac 12\left[N_L^{stuv} -  (-1)^{\so + \to} N^{-s-tuv}_L\right]\\
	\left.\av{\hat G^{\so + \to}_{LM} \hat C^{*, \uo + \vo}_{L' M'} }\right|_{\rm Gauss.} &=0
\end{split}}
\end{equation}
\subsection{QE responses}


Let the covariance of the CMB data respond as follows to a spin-$r$  ($r \ge 0$) anisotropy source $\alpha$:
\begin{equation}\label{eq:covresp}
	\delta  \av{_sX(\hn) \:_tX^*(\hn')} =   \sum_{\ell m, a = \pm r}\:_{a}\alpha(\hn) W_\ell^{a, st} \:_{s - a}Y_{\ell m}(\hn)  \:_{t}Y^*_{\ell m}(\hn')  +   W_\ell^{a, ts} \:_{s}Y_{\ell m}(\hn)  \:_{t-a}Y^*_{\ell m}(\hn')\:_{-a}\alpha(\hn')
\end{equation}
for some weights functions $W_\ell^{a, st}$. For instance, if the anisotropy can be described at the level of the CMB maps, such as for lensing, with
\begin{equation}
	_{s}\delta X(\hn) = \sum_{a = \pm r}\:_{a}\alpha(\hn) \left( \sum_{\ell m}\: R_\ell^{a, s} \:_sX_{\ell m} \Ylm {s- a}(\hn)\right)
\end{equation}
for harmonic responses $R$, then holds
\begin{equation}
	W_\ell^{a, st} = R^{a, s} C_\ell^{st}.
\end{equation}
However, Eq.~\eqref{eq:covresp} is more general.
Examples include:
\begin{itemize}
	\item Lensing: The source of anisotropy is the spin-1 field $_1\alpha(\hn)$, with response
	\begin{equation}
		\delta _sX(\hn) =  -\frac 12 \alpha_1(\hn) \eth _{s}X(\hn) - \frac 12 \alpha_{-1}(\hn) \bar \eth \:_sX(\hn) 
	\end{equation}
	where $\eth$ and $\bar \eth$ are the spin lowering and spin raising operator \JC{check notation } respectively. Hence
	\begin{equation}
		R_\ell^{1, s} =  ... R_\ell^{-1, s} =
	\end{equation}
	\item Modulation estimator: The source is spin 0, with response
	\begin{equation}
	\delta _sX(\hn) = \:_0\alpha(\hn) _{s}X(\hn)	
	\end{equation}
	Hence,
	\begin{equation}
		R_\ell^{st} = \delta_{st}
	\end{equation}
	\item Point sources in temperature:
	\begin{equation}
	W^{r, st}_\ell = \frac 14\delta_{r0}\delta_{s0}\delta_{t0} 
	\end{equation}
	\item Noise anisotropies (same as point sources but acting on beam deconvovled maps) \JC{does picking a fiducial noise value matter?}:
	\begin{equation}
	W^{r, st}_\ell = \frac 14\delta_{r0}\delta_{s0}\delta_{t0}  \frac{1}{b_\ell^2}
	\end{equation}
\end{itemize}

Let further the isotropic limit of the filtering procedure be the matrix $F$, defined through
\begin{equation}
	_{s}\bar X_{\ell m} = \sum_{s_2 = 0,2,-2}F_\ell^{s s_2} \:_{s_2}X_{\ell m} \quad \textrm{(isotropic approximation)}.
\end{equation} 
Then the gradient and curl responses of estimator~\eqref{QE} are
\begin{equation}\boxed{
	\begin{split}
		\resp^{gg}_L &= R_L^{st, r} + (-1)^r R_L^{st, -r}\\
		\resp^{cc}_L &= R_L^{st, r} - (-1)^r R_L^{st, -r} \\
		\resp^{gc}_L &= 0 = \resp^{cg}_L, \\
	\end{split}}
\end{equation}
where $R_L^{st, r}$ is
\begin{equation}
\begin{split}
R_L^{st, r} &= (-1)^{\so + \to }2\pi  \int_{-1}^1 d \mu\: d^L_{-\so - \to, -r}(\mu)\sum_{\tilde s_i,\tilde t_i = 0,2,-2}  \left[\xi^{\so \si \tilde s_i} (\mu)\psi^{\to \ti \tilde t_i \tilde s_i, r }(\mu) +  \xi^{\to \ti \tilde t_i}(\mu) \psi^{\so \si \tilde s_i \tilde t_i, r }(\mu) \right]
\end{split}
\end{equation}
with
\begin{equation}
\begin{split}
\xi^{\so\si \tilde s_i}(\mu) &\equiv  \sum_\ell \left(\frac{2\ell + 1}{4\pi}\right)w^{\so\si}_\ell F_\ell^{\si \tilde \si} d^\ell_{\so,\tilde \si}(\mu) \\
\psi^{\so\si \tilde \si \tilde \ti, r}(\mu) &\equiv  (-1)^r \sum_\ell \left(\frac{2\ell + 1}{4\pi}\right)w^{\so \si}F^{\si \tilde \si}_\ell W_\ell^{-r, -\tilde \ti \tilde \si} d^\ell_{\so,-\tilde \ti + r}(\mu) 
\end{split}
\end{equation}

\subsection{Optimal QE weights}Optimal QE weights are easily gained from the representation in Eq.~\ref{eq:covresp} of the anisotropy. Let
\begin{equation}
	_{\pm r}\hat g^\alpha(\hn) = \frac{\delta }{\delta _{\mp r}\alpha (\hn)} -\frac 12\: _{s_1}X^{\rm dat} \Cov^{-1}_{s_1s_2} \:_{s_2}X^{\rm dat}.
\end{equation}
where $\Cov_{s_1 s_2}(\hn, \hn') \equiv \av{_{s_1}X^{\rm dat}(\hn) \:_{s_2}X^{\rm dat}(\hn') }$
We find
\begin{equation}\boxed{
	w_\ell^{st} = \delta_{st} \textrm{   (1st leg)  } \quad 	w_\ell^{s + r, t} = W^{-r, st}_\ell \textrm{   (2nd leg)  }}
\end{equation}
\end{document}